\clearpage
\chapter{\textbf{Grundlagen}}\label{grundlagen}
%\addtocontents{toc}{\vspace{0.8cm}}

\section{Unterkapitel}\label{unterkapitel}
\addtocontents{toc}{\vspace{0.8cm}}

Wir sehen im Folgenden die Formel für die Faltung von Wahrscheinlichkeitsdichtefunktionen, als Gleichungsarray:

\begin{align}
(p_i * p_j)(n) & =  \sum_{k \in \mathbb{D}} p_i(k) \cdot p_j(n - k) \\
p_{total} & =  p_0 \ast p_1 \ast \ldots p_{n-1}; \forall n
\end{align}

% Formel
Hier ist nur eine einfache Formel mit der \texttt{equation}-Umgebung für die Minkowski Metrik:
\begin{equation}\label{Minkowski}
D\left(X,Y\right)=\left(\sum_{i=1}^n |x_i-y_i|^p\right)^{1/p}\\
\end{equation}

Wie in Gleichung \ref{Minkowski} zu erkennen ist, ergibt sich die L2-Norm (Euklidische Distanz), wenn man den Exponenten $p = 2$ wählt.

Support Vector Machines \cite{Haykin99} nutzen die Euklidische Distanz (oder äquivalent) das Skalarprodukt.
%% Zwei Abbildungen, die zusammen gehören

%\begin{figure}
%        \centering
%        \begin{minipage}[c]{0.45\textwidth}
%                \includegraphics[height=6.5cm]{pic/dateiname1.png}
%        \end{minipage}
%        \begin{minipage}[c]{0.45\textwidth}
%                \includegraphics[height=6.5cm]{pic/dateiname2.png}
%        \end{minipage}
%        \caption{Zwei Abbildungen}\label{fig:zwei_abb}
%\end{figure}
