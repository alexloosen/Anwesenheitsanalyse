\clearpage
\chapter{\textbf{Einleitung}}\label{einleitung}
%\addtocontents{toc}{\vspace{0.8cm}}


%\par\medskip

Gebäudeautomatisierung bezeichnet die automatische Steuerung und Regelung von Ge-bäudetechnik 
wie Heizung, Lüftung oder Beleuchtung. Während sie bisher hauptsächlich für die Optimierung 
der Energieeffizienz von gewerblichen und öffentlichen Gebäuden genutzt wurde, welche im 
Zuge solcher Optimierungsschritte als ,,Smart Buildings'' bezeichnet werden, 
rückte sie im Verlauf der letzten Jahren zunehmend unter dem Begriff ,,Smart Home'' auch in den privaten 
Bereich.\\
Die beiden Begriffe stehen in den letzten Jahren so im Vordergrund, weil eine
Verbesserung der Energieeffizienz durch bauphysikalische Maßnahmen, wie verminderte 
Wärmeverluste durch bessere Isolation, an ihre Grenzen gestoßen sind.
\\\\
In Deutschland haben private Haushalte einen Anteil am gesamten Energieverbrauch des Landes von 
etwa 29\%\footnote[1]{\cite{Umweltbundesamt1}}. Davon nimmt die Erzeugung der Raumwärme einen Anteil
von etwa 68\%\footnote[2]{\cite{Umweltbundesamt}} ein. Vor allem bezogen
auf den Energieverbrauch durch das Heizen stellt die Gebäudeautomatisierung also einen der größten
Sektoren dar, in denen eine Optimierung der Prozesse entscheidend wäre. Beispielsweise können Wärmeverluste
über die Nacht hinweg um 20\% verringert werden, wenn abends die Rolläden heruntergelassen werden.\\\\
Zur weiteren Steigerung der Energieeffizienz ist es also nötig, die Gebäudetechnik insofern
automatisch anzusteuern, sodass sog. Performance-Gaps vermieden werden. Performance-Gaps
stellen eine Diskrepanz im Energieverbrauch eines Gebäudes zwischen einem theoretischen 
Soll-Wert zu einem tatsächlichen Ist-Wert dar.\\


\section{Motivation und Aufgabenstellung}
%\addtocontents{toc}{\vspace{0.8cm}}

Für nahezu alle Bereiche der Gebäudeautomatisierung stellt die Anwesenheit 
von Personen eine zentrale Variable dar. Es ist entscheidend bei einem Optimierungsprozess zu erkennen, zu
welchen Zeitpunkten sich Menschen in einem Raum aufhalten. Beispielsweise werden in vielen gewerblich 
genutzen Gebäuden werden grundsätzlich alle Räume beheizt - unabhängig davon, ob Personen tatsächlich anwesend sind. 
Eine Erkennung von menschlicher Anwesenheit ist bei der Optimierung also essenziell.\\\\
Da die direkte Messung von Anwesenheit über z.B. Infrarot- oder Kamerasensoren rechtlich problematisch ist, 
soll in dieser Arbeit untersucht werden, inwiefern Machine-Learning Algorithmen genutzt werden können, 
um menschliche Präsenz anhand von CO2-, Temperatur und Luftdruckwerten der Raumluft festzustellen.\\\\
Die Motivation der Optimierung der Gebäudeautomatisierung exisitert, da ein steigender CO2-Gehalt der 
Raumluft nachweislich mit einer Abnahme der menschlich kognitiven Leistung einhergeht. Mehrere Studien
konnten belegen, dass sowohl sprachliche als auch logisch-mathematische Fähigkeiten abnehmen, sobald 
der CO2-Gehalt der  Raumluft bestimmte Werte überschreitet.\\ 
Um eine angemessene Datengrundlage zu schaffen, wurden in diversen Büro-Räumen der FH Aachen Temperatur-,
Luftfeuchtigkeits-, Infrarot- und CO2-Sensoren angebracht, deren Messungen kontinuierlich auf einer Datenbank
gespeichert wurden. Der Zeitraum der Messwerte begann Mitte 2021.\\
In allen Räumen sind täglich ein oder mehrere
Personen im Rahmen eines ca. 8-stündigen Arbeitstages anwesend, weshalb die Temperatur-, Luftfeuchtigkeits-
und CO2-Messwerte über einen Tag hinweg Schwankungen aufweisen. Diese sollen von einem Machine Learning Modell
mit der direkten Präsenzmessung des Infrarotsensors gegenübergestellt werden, sodass das fertig trainierte Modell
anhand der Messwerte von Temperatur, Luftfeuchtigkeit und CO2 Aussagen über menschliche Präsenz treffen kann.\\
Die Frage, ob ein in einem Raum trainiertes Modell auch präzise Aussagen über menschliche Anwesenheit in einem anderen Raum
treffen kann, stellt eine besondere Bedeutung dar.\\\\
Es gab keine Einschränkungen hinsichtlich dessen, welche konkreten Machine-Learning Algorithmen 
benutzt werden sollen.\\
Als Programmiersprache für das Projekt wurde Python gewählt. Python ist wegen umfassender
Machine-Learning Bibliotheken und einfacher Auswertungstechniken anhand von z.B. Graphen und Statistiken 
für diesen Anwendungsfall gut geeignet.



%% Beispiel für das Einfügen einer Abbildung

%\begin{figure}[h]
%	\centering
%		\includegraphics[width=0.8\textwidth]{pic/dateiname.png}
%	\caption{Beispielbild}
%	\label{fig:beispielbild}
%\end{figure}
%\vspace{7cm} % Abstand unter dem Bild


%\newpage

\section{Vorgehensweise}
%\addtocontents{toc}{\vspace{0.8cm}} % -> Abstand im Inhaltsverzeichnis

% Untersuchungsverlauf(pro Kapitel ein kurzer Absatz mit Verweis auf die Kapitelnummer)

Das Projekt beschäftigt sich im Schwerpunkt mit den folgenden Arbeitsschritten:
\begin{itemize}
    \item Datenbeschaffung durch Datenbankzugriffe per SQL
    \item Analyse und Vorbereitung der Daten (Pre-Processing)
    \item Trainieren von Machine-Learning Modellen anhand der vorbereiteten Datensets
    \item Ergebnisauswertung durch Gegenüberstellung verschiedener Datensets und Modellen
\end{itemize}
Da es zwischen allen verfügbaren Datensets der einzlnen Räume und Machine-Learning-Modellen eine Vielzahl an
Kombinationsmöglichkeiten gibt, ist die Projektarbeit mit dem Anspruch angelegt, ein möglichst übersichtliches, 
gut gekapseltes Python Programm zu erstellen, das einfach und schnell verschiedene Datensets  
verarbeiten und mit einer dem Forschungszweck angemessenen Anzahl von Machine-Learning Modellen auszuwerten. \\\\
Um einen Vergleich der Ergebnisse zu ermöglichen, sollen diese klar und verständlich dargestellt werden. 
Da nicht bei allen Algorithmen die gleichen Leistungsindikatoren genutzt werden, sollen hauptsächlich nur 
jene Indikatoren betrachtet werden, die bzgl. aller Algorithmen auch gleiche Bedeutung haben. Falls 
modellspezifische Leistungsindikatoren als besonders erkenntnisreich erachtet werden, wird dies in dieser Arbeit
angemerkt.