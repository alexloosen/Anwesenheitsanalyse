\clearpage
\chapter{\textbf{Zusammenfassung und Ausblick}}\label{zusammenfassung}
%\addtocontents{toc}{\vspace{0.8cm}}

\section{CO2 als Anwesenheitsindikator}
Die Genauigkeit aller Modelle bei der Präsenzerkennung während der gesamten Durchführungsphase des Projektes 
lag zwischen 75\% und 94\%.
Hierbei wiesen Decision Tree Modelle für diesen Anwendungsfall die höchsten Genauigkeiten auf. 
Die Beziehung zwischen der aktuellen Tageszeit und dem CO2-Gehalt der Luft erwies sich für das Training von 
Machine Learning Algorithmen als sehr nützlich. 
Resultate vorangegangener Forschung unterstützen dieses Ergebnis ebenfalls\footnote[1]{Bsp. \cite{IPPR}}.\\
Weiterhin konnte gezeigt werden, dass alle Modelle zusätzlich in der Lage sind, menschliche Präsenz außerhalb
der trainierten Uhrzeiten korrekt zu erkennen, wenn die Modelle mit Deltas zu Vergangenheitswerten des 
CO2-Gehalts trainiert werden. Die Präsenzerkennung ist also vollständig unabhängig von den Umständen des
Trainingssets und kann sowohl für andere Räume, als auch zu anderen Tageszeiten effektiv genutzt werden. 
Der CO2-Gehalt der Luft kann somit als geeigneter Indikator für menschliche Präsenz in Innenräumen angesehen 
werden.\\\\
Zudem wurde gezeigt, dass der Einsatz von CO2-Sensoren bei Gebäudeautomatisierungssystemen zweckmäßig ist.
Ein weiterer Vorteil besteht darin, dass diese kostengünstig und leicht zu implementieren sind.


\section{Mögliche Verbesserungen}
Die Genauigkeit von Machine Learning Modellen kann allgemein durch die Hinzugabe von mehr Daten in ein Modell 
verbessert werden. \\
Da der Datenbestand, auf die sich diese Arbeit stützt, über die Bearbeitungszeit des Projektes 
kontinuierlich vergrößert wurde, wuchs damit auch die Anzahl an Anhaltspunkten für Zusammenhänge zwischen den 
Datenfeldern stetig an. Besonders nützlich wäre das Sammeln von Messdaten über ein oder mehrere Jahre hinweg.
Somit hätten die Modelle eine Chance, Gewohnheiten in z.B. Urlaubstagen, Feiertagen oder auch wiederkehrenden
Meetings korrekt einzuordnen, sodass Gebäudeautomatisierungssysteme ein Maximum an Komfort und Energieeffizienz
herstellen können.

\newpage

Die Anwendung eines Modells auf andere Räume lies die Vermutung zu, dass diese Art von Präsenzerkennung das 
Potential aufweist, anhand eines Traningssets auch eine Vielzahl anderer Datensets klassifizieren zu können.
Durch Aufzeichnung von Daten aus anderen Räumen könnte diese Vermutung genauer untersucht werden.  \\\\
Da Infrarotsensoren inhärent nicht in der Lage sind kontinuierliche Präsenz zu erkennen, wenn sich die Personen
im Raum nicht bewegen, kam es innerhalb des Datensets immer wieder zu kleinen Messfehlern, die durch Gruppierung
und Durchschnittsberechnung der Präsenzwerte behoben werden mussten. Es ist klar davon auszugehen, dass eine 
genauere Datenlage auch die Qualität der trainierten Modelle steigern würde.\\\\
Zusätzlich existierten im Datenset durch seltene Probleme mit der Hardware längere Datenreihen mit falschen 
Labelwerten, welche während der Bearbeitung ausgeschlossen werden mussten. Da sich dadurch die für das Training 
geeignete Datenmenge verringerte, ist auch hier davon auszugehen, dass das Aussortieren dieser Werte
mit einer Verringerung der Modellqualität einherging. 
