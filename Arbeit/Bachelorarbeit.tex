\documentclass[letterpaper,12pt]{article}

%\include{formatting}   % stick all of the LaTeX formatting into a separate file

\usepackage{amsmath}
\usepackage{graphicx}
\usepackage{hyperref}
\usepackage[utf8]{inputenc}
\usepackage{hyphenat}
\usepackage[ngerman]{babel}
\title{Prognose der Anwesenheit von Personen für die Gebäudeautomatisierung 
mittels Umweltsensordaten}
\author{Alexander Loosen}
\date{14.02.2022}
\begin{document}
\maketitle
\thispagestyle{empty}
\pagebreak
\section{Einleitung}
Gebäudeautomatisierung bezeichnet die automatische Steuerung und Regelung von Gebäudetechnik 
wie Heizung, Lüftung oder Beleuchtung. Während sie bisher hauptsächlich für die Optimierung 
der Energieeffizienz von gewerblichen und öffentlichen Gebäuden genutzt wurde, welche in 
Zuge solcher Optimierungsschritte als ,,Smart Buildings'' bezeichnet werden, 
rückt sie in den letzten Jahren zunehmend unter dem Begriff ,,Smart Home'' auch in den privaten 
Bereich. Die beiden Begriffe stehen in den letzten Jahren so im Vordergrund, da eine
Verbesserung der Energieeffizienz durch bauphysikalische Maßnahmen, wie verminderte 
Wärmeverluste durch bessere Isolation, an ihre Grenzen gestoßen sind.
\\
Zur weiteren Steigerung der Energieeffizienz ist es also nötig, die Gebäude-technik
automatisch anzusteuern, sodass sog. Performance-Gaps vermieden werden. Performance-Gaps
stellen eine Diskrepanz im Energieverbrauch eines Gebäudes zwischen einem theoretischen 
Soll-Wert zu einem tatsächlichem Ist-Wert dar.
Für nahezu alle Bereiche der Gebäudeautomatisierung stellt die Anwesenheit 
von Personen eine zentrale Variable dar. Da die direkte Messung von Anwesenheit über z.B. 
Infrarotsensoren nicht verlässlich ist, soll in dieser Arbeit untersucht werden, inwiefern 
Machine-Learning Algorithmen genutzt werden können, um eine genaue Erwartung über 
die Anwesenheit von Personen anhand von CO2-Werten in der Raumluft zu treffen.

\end{document}