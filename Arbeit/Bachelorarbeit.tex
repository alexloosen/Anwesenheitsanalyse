\documentclass[letterpaper,12pt]{article}

%% --- THIS MAKES EQUATIONS BETTER ---
%
\usepackage{mathtools}               				   % mathtools and amsmath


% ---------- CHOOSE A FONT ----------
%
\usepackage[protrusion=true,expansion=true]{microtype} % Better typography
\usepackage[T1]{fontenc}                               % Better typography
\usepackage{lmodern}

\usepackage[scaled=0.92]{helvet}                       % load Helvetica font
\renewcommand*\familydefault{\sfdefault}              % Helvetica font for main text

%\usepackage{times}									   % Times font for main text
%\usepackage{txfonts}								   % Equations using Times-like font


% (or leave all font commands above commented out for LaTeX default, Computer Modern)


\usepackage[english]{babel}							   % hyphenation etc for English


% --------- MISC FORMATTING ---------
%
\usepackage{setspace}                % needed for doublespacing
\doublespacing                       % doublespaced line spacing
\usepackage{natbib}                  % for our chosen bibliography style
\usepackage{graphicx}                % for importing graphics into figures
\usepackage{float}                   % better control of figure placement
\usepackage{hyperref}                % for clickable URLs and email addresses
\urlstyle{sf}                        % san-serif font for URLs
\usepackage[margin=1.2in]{geometry}  % control page margins
\usepackage[short]{datetime}         % precise date/time stamp on titlepage
\usepackage[labelfont=bf]{caption}   % make caption labels boldface
\usepackage[bottom]{footmisc}        % footnotes at bottom of page
\setlength{\skip\footins}{10mm}      % obsessing about footnote spacing
\setlength{\parskip}{1ex}            % space between paragraphs
%\setlength{\parindent}{3em}	     % paragraph indentation
%\usepackage{lineno}                  % add line numbers to margin
%\def\linenumberfont{\normalfont\footnotesize\sffamily} % line numbers
%\setlength\linenumbersep{9mm}                          % line numbers
%\linenumbers                                           % line numbers
\usepackage{authblk}                 % author and affiliation formatting
\renewcommand\Affilfont{\small}

% --------- SECTION HEADINGS ---------
%
\usepackage[compact]{titlesec}
\titleformat*{\section}{\sffamily\normalsize\bfseries\uppercase}
\titlespacing*{\section}{0pt}{1.5ex}{0ex}
\titleformat*{\subsection}{\sffamily\normalsize\bfseries}
\titlespacing*{\subsection}{0pt}{0ex}{0ex}
\titleformat*{\subsubsection}{\sffamily\normalsize\itshape}
\titlespacing*{\subsubsection}{0pt}{0ex}{0ex}

% ------- CUSTOM TITLE FORMAT -------
%
\makeatletter
\renewcommand{\maketitle}{
\begin{flushleft}       % right align
\vspace*{5mm}
\MakeUppercase{\Large\sffamily\bfseries\@title}   % increase the font size of the title
%\rule{\textwidth}{0.5pt}
\vspace{15mm}\\         % vertical space between the title and author name
{\normalsize\sffamily\@author}        % author name
\end{flushleft}
}
\makeatother

% ------- DEGREE SYMBOL ---------
%
\newcommand{\degrees}{\ensuremath{^\circ}} % for nice degree symbol

% ------- DUMMY TEXT --------
%
\usepackage{lipsum}         % stick all of the LaTeX formatting into a separate file

\usepackage{amsmath}
\usepackage{graphicx}
\usepackage{hyperref}
\usepackage[utf8]{inputenc}
\usepackage{hyphenat}
\usepackage[ngerman]{babel}
\title{Prognose der Anwesenheit von Personen für die Gebäudeautomatisierung 
mittels Umweltsensordaten}
\author{Alexander Loosen}
\date{14.02.2022}
\begin{document}
\maketitle
\thispagestyle{empty}
\pagebreak
\section{Einleitung}
Gebäudeautomatisierung bezeichnet die automatische Steuerung und Regelung von Gebäudetechnik 
wie Heizung, Lüftung oder Beleuchtung. Während sie bisher hauptsächlich für die Optimierung 
der Energieeffizienz von gewerblichen und öffentlichen Gebäuden genutzt wurde, welche in 
Zuge solcher Optimierungsschritte als ,,Smart Buildings'' bezeichnet werden, 
rückt sie in den letzten Jahren zunehmend unter dem Begriff ,,Smart Home'' auch in den privaten 
Bereich. Die beiden Begriffe stehen in den letzten Jahren so im Vordergrund, da eine
Verbesserung der Energieeffizienz durch bauphysikalische Maßnahmen, wie verminderte 
Wärmeverluste durch bessere Isolation, an ihre Grenzen gestoßen sind.
\\
Zur weiteren Steigerung der Energieeffizienz ist es also nötig, die Gebäude-technik
automatisch anzusteuern, sodass sog. Performance-Gaps vermieden werden. Performance-Gaps
stellen eine Diskrepanz im Energieverbrauch eines Gebäudes zwischen einem theoretischen 
Soll-Wert zu einem tatsächlichem Ist-Wert dar.
Für nahezu alle Bereiche der Gebäudeautomatisierung stellt die Anwesenheit 
von Personen eine zentrale Variable dar. Da die direkte Messung von Anwesenheit über z.B. 
Infrarotsensoren nicht verlässlich ist, soll in dieser Arbeit untersucht werden, inwiefern 
Machine-Learning Algorithmen genutzt werden können, um eine genaue Erwartung über 
die Anwesenheit von Personen anhand von CO2-Werten in der Raumluft zu treffen.

\end{document}